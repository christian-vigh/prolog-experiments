% LaTeX Macros for Natural Deduction Categorial Grammar
% =====================================================
% Bob Carpenter
% May 5, 1994

% This file contains the LaTeX macros necessary to generate
% natural deduction style CG proofs.  All of them may be easily
% modified to change the look of the output derivations.  They're
% presented in this file top down from the highest level to the 
% lowest

% Derivations
% ----------------------------------------------------------------------

% axr{Result}{Rule}
% ----------------------------------------------------------------------
% sets axiomatic rule with no antecedent, result Result, and rule Rule
% ----------------------------------------------------------------------
\newcommand{\axr}[2]{\begin{array}[t]{c}
                         \hrulefill\makebox[0pt][l]{$#2$}
                         \\
                         #1
                       \end{array} }

% \infr{Antecedent}{Result}{Rule}
% ----------------------------------------------------------------------
% sets inference scheme with antecedent Antecedent, result Result
% and rule Rule
% ----------------------------------------------------------------------
\newcommand{\infr}[3]{
  \begin{array}[t]{@{}c@{}}
    #1
    \\[-4.5pt]
   \hrulefill\makebox[0pt][l]{$#3$} 
    \\ 
    #2
\end{array}}


% Categories
% ----------------------------------------------------------------------

% \synsem{Syn}{Sem}
% ----------------------------------------------------------------------
% sets category with syntax Syn and semantics Sem
% ----------------------------------------------------------------------
\newcommand{\synsem}[2]{#1\/\colon #2}


% Semantic Terms -- Lambda Calculus
% ----------------------------------------------------------------------

% \lam{Var}{Scope}
% ----------------------------------------------------------------------
% sets lambda abstraction of Var over Scope
% ----------------------------------------------------------------------
\newcommand{\lam}[2]{\lambda #1 . #2}

% \con{C}
% ----------------------------------------------------------------------
% sets semantic constant C
% ----------------------------------------------------------------------
\newcommand{\con}[1]{{\bf #1}}


% Logic
% ----------------------------------------------------------------------

% \conj, \disj, \neg
% ----------------------------------------------------------------------
% sets logical operators of conjunction and disjunction and negation
% (negation built in)
% ----------------------------------------------------------------------
\newcommand{\conj}{\wedge}
\newcommand{\disj}{\vee}

% \setlist{X}
% ----------------------------------------------------------------------
% sets X in set braces
% ----------------------------------------------------------------------
\newcommand{\setlist}[1]{\{ #1 \}}


% Syntactic Terms - Categorial Grammar
% ----------------------------------------------------------------------

% \fd{A}{B}, \bk{A}{B}, \bang{A}{B}, \scop{A}{B}
% ----------------------------------------------------------------------
% sets A/B, A\B, A-B and scop(A,B) respectively
% ----------------------------------------------------------------------
\newcommand{\fd}[2]{#1 / #2}
\newcommand{\bk}[2]{#1 \backslash #2}
\newcommand{\bang}[2]{#1 {\uparrow} #2}
\newcommand{\scop}[2]{#1{\Uparrow}#2}


% Expressions
% ----------------------------------------------------------------------

% \men{W}
% ----------------------------------------------------------------------
% sets mention of word W
% ----------------------------------------------------------------------
\newcommand{\men}[1]{\mbox{{\sl #1}}}


% Deduction Scheme Names
% ----------------------------------------------------------------------
% sets the various deduction scheme names
% ----------------------------------------------------------------------
\newcommand{\fel}{\elim{/}}
\newcommand{\bel}{\elim{\backslash}}
\newcommand{\fin}[1]{\introindex{/}{#1}}
\newcommand{\bin}[1]{\introindex{\backslash}{#1}}
\newcommand{\gin}[1]{\introindex{{\uparrow}}{#1}}
\newcommand{\scopr}[1]{\elim{\Uparrow}^{#1}}
\newcommand{\scin}{\intro{{\Uparrow}}}
\newcommand{\coel}{\elim{c}}
\newcommand{\qi}{\intro{{\Uparrow}}}
\newcommand{\qpush}[1]{\scopr{#1}}
\newcommand{\qpop}[1]{\scopr{#1}}
\newcommand{\qqpush}[1]{\elim{\mbox{\small\it q}}^{#1}}
\newcommand{\qqpop}[1]{\elim{\mbox{\small\it q}}^{#1}}
\newcommand{\lexr}{l}
\newcommand{\empr}{\mbox{$\nullstring$}}
\newcommand{\id}{i}
\newcommand{\cut}{c}
\newcommand{\der}{d}
\newcommand{\coor}{Co}
\newcommand{\nbc}{nbc}
\newcommand{\extr}{ex}
\newcommand{\topc}{tp}


